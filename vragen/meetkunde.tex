\documentclass[12pt]{article}

\textwidth 15cm \textheight 23cm \evensidemargin 1cm
\oddsidemargin 1cm \topmargin -1cm \topskip 0cm
\parindent 0pt
\parskip 0cm
%\parskip \medskipamount


\usepackage[dutch]{babel}
\usepackage{amssymb,amsthm,amsmath}
\usepackage[utf8]{inputenc}
%\usepackage{nopageno}
\usepackage{pdfpages}
\usepackage{enumerate}
\usepackage{caption}
\usepackage{wrapfig}
\usepackage{pgf,tikz}
\usepackage{color}
\usetikzlibrary{arrows}
\usetikzlibrary{patterns}
\usepackage[version=3]{mhchem}
\usepackage{multicol}
\setlength\columnseprule{.4pt}
\usepackage{array}

\newcommand{\degree}{\ensuremath{^\circ}}
\reversemarginpar

\newcommand{\cm}{\mbox{ cm}}

\newcounter{punten}
\setcounter{punten}{0}
\newcounter{nvraag}
\setcounter{nvraag}{1}
\newcommand{\vraag}[1]{\vspace*{20pt}{\large\bf Vraag \arabic{nvraag} \addtocounter{nvraag}{1}}\marginpar{\vspace*{-10pt}\fbox{\parbox{1.5cm}{{\Large\color{black!45}{#1}}\vspace*{1cm}\hspace*{0.75cm}}}}\addtocounter{punten}{#1}}

\newcounter{noef}
\setcounter{noef}{1}
\newcommand{\oefening}{\vspace*{20pt}{\large\bf Oefening \arabic{noef} \addtocounter{noef}{1}}}

\newcommand{\dotrule}[1]{%
   \parbox[t]{#1}{\vspace*{0.05cm}\dotfill}}

\newcommand{\dotlines}[1]{   
\foreach \n in {1,...,#1}{

\vspace*{0.05cm}
\dotfill
}}

\newcommand{\ruitjes}[1]{
\definecolor{cqcqcq}{rgb}{0.65,0.65,0.65}
\begin{tikzpicture}[scale=1.01,line cap=round,line join=round,>=triangle 45,x=1.0cm,y=1.0cm]
\draw [color=cqcqcq,dash pattern=on 1pt off 1pt, xstep=0.5cm, ystep=0.5cm] (0,-#1) grid (15,0);
\end{tikzpicture}
}

\newcommand{\assenstelsel}[5][1]{
\definecolor{cqcqcq}{rgb}{0.65,0.65,0.65}
\begin{tikzpicture}[scale=#1,line cap=round,line join=round,>=triangle 45,x=1.0cm,y=1.0cm]
\draw [color=cqcqcq,dash pattern=on 1pt off 1pt, xstep=1.0cm,ystep=1.0cm] (#2,#4) grid (#3,#5);
\draw[->,color=black] (#2,0) -- (#3,0);
\draw[shift={(1,0)},color=black] (0pt,2pt) -- (0pt,-2pt) node[below] {\footnotesize $1$};
\draw[color=black] (#3.25,0.07) node [anchor=south west] { x};
\draw[->,color=black] (0,#4) -- (0,#5);
\draw[shift={(0,1)},color=black] (2pt,0pt) -- (-2pt,0pt) node[left] {\footnotesize $1$};
\draw[color=black] (0.09,#5.25) node [anchor=west] { y};
\draw[color=black] (0pt,-10pt) node[right] {\footnotesize $0$};
\end{tikzpicture}
}

\newcommand{\visgraad}[1]{\begin{tabular}{p{0.5cm}|p{#1}}&\\\hline\\\end{tabular}}


% geef tabular iets meer ruimte
\setlength{\tabcolsep}{12pt}
\renewcommand{\arraystretch}{1.5}


\begin{document}

\section{Vectoren}

\oefening

Welke gelijkheden (omcirkel één of meer) zijn waar? $A$ en $B$ zijn twee verschillende punten.
\begin{enumerate}[(A)]
  \item $\vec{AB} = -\vec{AB}$
  \item $|\vec{AB}| = -|\vec{AB}|$
  \item $\vec{AB} = -\vec{BA}$
  \item $|\vec{AB}| = -|\vec{BA}|$
\end{enumerate}

\oefening
Als er geldt dat $\vec{AB}+\vec{BC}=\vec{AC}$, welke gelijkheid of ongelijkheid (omcirkel één) geldt dan altijd:
\begin{enumerate}[(A)]
  \item $|AB|+|BC|=|AC|$
  \item $|AB|+|BC|<|AC|$
  \item $|AB|+|BC|\le|AC|$
  \item $|AB|+|BC|>|AC|$
  \item $|AB|+|BC|\ge|AC|$
\end{enumerate}

\end{document}



















%%%%%%%%%%%%%%%%%%%%%%%%%%%%%%%%%%%%%%%%%%%%%

Een vraag begin je met \verb#\vraag{PUNTEN}# zoals

\vraag{5} Los op in $\mathbb{H}$:
\begin{enumerate}[(a)]
  \item $ij-kx=0$
\end{enumerate}

Ruitjes met \verb#\ruitjes{Xcm}# zoals\\
\ruitjes{2cm}

Een aantal lijntjes met \verb#\dotlines{N}# zoals
\dotlines{2}\\

Een lijntje verder aanvullen met \verb#\dotfill# zoals \dotfill\\

Een beetje blanco ruimte voorzien met \verb#\vspace*{Xcm}# zoals
\vspace*{2cm}

Een stukje in twee verdelen met \verb#\begin{minipage}\end{minipage}# zoals

\begin{minipage}[c]{0.5\textwidth}
\centering
Links
\end{minipage}
\begin{minipage}[c]{0.5\textwidth}
\centering
Rechts
\end{minipage}
\vspace*{0.5cm}



Een figuur invoegen met \verb#\includegraphics[width=0.5\textwidth]{PATH}#\\

Het lopend totaal van de punten weergeven met \verb#\arabic{punten}# zoals in de kantlijn.
\marginpar{\fbox{\tiny \arabic{punten}}}

%\newpage
%\appendix
%\section*{Bijlagen}
%\vspace*{-0.25cm}
%\includegraphics[width=0.9\textwidth]{pse}


















